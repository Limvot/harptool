\documentclass[10pt,letterpaper]{article}

\usepackage[margin=1in]{geometry}
\usepackage{graphicx}
\title{HarpTool Readme}
\author{Chad D. Kersey, Nathan Braswell}

\begin{document}
\maketitle
\section*{Disclaimer}
This document, like the work it documents, is very much a work in progress.
Send any corrections, updates, suggestions, or complaints (preferably in patch form, this file is under \texttt{harptool/doc} in the HarpTool repo) to \texttt{cdkersey@gatech.edu}.

\section*{Introduction}
HARP is two things, a multi-year Heterogeneous Architecture Research Project, and an implementation of a specific Heterogeneous Architecture Research Prototype. This document is an introduction to HarpTool, not an assembly reference. For the way the assembly actually works, please see the other document harp\_iset.

\section*{HarpTool}
The assembler/linker/emulator/disassembler program for HARP is called HarpTool.
It is a multiple-function executable, its function selected with the first command line argument.
When run with no command line arugments the HarpTool executable prints a help message explaining the available command line arguments.

All of the HARP utilities can take an archID as a command line parameter.
If none is provided, a default will be assumed.

\section*{Contents}

\begin{itemize}
  \item{Building and Installing}
  \item{Tests and Examples}
  \item{Building and Running Examples}
  \item{Architecture Identifier String}
  \item{Assembly Syntax}
  \item{Lanes, Branches}
  \item{Some Conventions}
\end{itemize}

\section*{Building and Installing}
Building and installing HarpTool is done in a standard unix way - make and (sudo) make install from the src directory.

\section*{Tests and Examples}
Tests and examples can be found under the src/test directory. Running make in this directory will assemble and link all of the test programs.

\section*{Building and Running Examples}
run them with \begin{verbatim}
  harptool -E -c example.bin
\end{verbatim}
(That's -E for emulate and -c to specify the input file)
How to use HarpTool to assemble and link HARP assembly can be seen from looking at output of building some of the tests
\begin{verbatim}
../harptool -A -o vector_dot.HOF vector_dot.s
\end{verbatim}
Assemble a .s assembly file to a .HOF (HARP object file)
\begin{verbatim}
../harptool -L -o vector_dot.bin boot.HOF lib.HOF vector_dot.HOF
\end{verbatim}
Sink several HARP object files into a .bin file that can be run with the emulator (as above)\\
\\
It can also use the --arch parameter to modify what arch it is assembled and linked for.
\begin{verbatim}
../harptool -A --arch 4b16/16/2 -o vector_dot.4b.HOF vector_dot.s
../harptool -L --arch 4b16/16/2 -o vector_dot.4b.bin boot.4b.HOF lib.4b.HOF vector_dot.4b.HOF
\end{verbatim}

\section*{Pitfalls and Gotchas}
\begin{itemize}


\item{Be careful to not use opcodes that do not exist - as of right now HarpTool will silently convert them to noops. This can be the source of hard to find bugs.}
\item{Be careful to note that the main lane of execution is lane 0 when doing SIMD execution, and that it will retain the register state it had during SIMD execution when the lanes rejoin.}
\end{itemize}

\section*{Architecture Identifier String (\texttt{ArchID})}
The best way to understand the multifaceted parameterizablity of the HARP ISAs is to study the architecture identifier strings used to uniquely identify a single HARP instruction set architecture.
We'll start by breaking down Harptool's default \texttt{ArchID}: \texttt{8w32/32/8}:

\begin{center}
\begin{tabular}{rl}
\textbf{Field}&\textbf{Meaning}\\
\hline
\texttt{8} &8-byte (64-bit) registers and addresses\\
\texttt{w} &Word-based (64-bit) fixed-width instruction encoding\\
\texttt{32}&32 general-purpose registers per lane\\
\texttt{32}&32 predicate registers per lane\\
\texttt{8} &8 SIMD lanes\\
\end{tabular}
\end{center}

All ArchIDs have a similar format, although the SIMD lanes field can be omitted, as object files are still fully compatible even if the number of lanes changes. 



\section*{Assembly Syntax}
The assembler converts assembly files to HOF, the Harp Object Format.

The assembly language is fairly easy to pick up from the HarpTools examples. It is RISC-like, and written destination register first (in this it differs from Unix assembly syntax).
Lines are ended by a semicolon. Comments are done C-style, like /* this */
Registers names are prefixed with the percent sign (\%) and predicate register names with the at symbol (@).
Predicated instructions are prefixed with the predicate register name and a question mark:
\begin{verbatim}
  @p0 ? addi %r7, %r1, #1
\end{verbatim}
A small set of directives is provided to express non-instruction data:

\begin{center}
\begin{tabular}{cl}
\textbf{Directive}&\textbf{Use}\\
\hline
\texttt{.align 256}    &Align next symbol to a multiple of 256 bytes.\\
\texttt{.word 0x1234}  &Insert a word with the value \texttt{0x1234}.\\
\texttt{.byte 0xff}    &Insert a byte with the value \texttt{0xff}.\\
\texttt{.def SYM 123}  &Replace SYM with 123 in immediate operands.\\
\texttt{.entry}        &Make the next label the HOF executable entry point.\\
\texttt{.global}       &Give the next label global (external) linkage.\\
\texttt{.perm rw}      &Set HOF permissions of the next label to read/write.\\
\texttt{.string "Str"} &Create a null terminated string.\\
\end{tabular}
\end{center}

\section*{SIMD Operation}
The HARP ISAs are inherently SIMD.
In addition to designs with a single set of functional units and architectural registers, designs are allowed that replicate these while retaining a single front-end and memory system.
This allows for multiple threads executing the same stream of instructions to simultaneously occupy multiple ``lanes'' of the processor.
When a predicated control flow instruction occurs without unanimous agreement among predicate registers, a divergent branch has occurred.
The current response to this is to trap to the operating system (interrupt number 4).

\subsection{Instructions for SIMD Operation}
The \texttt{clone}, \texttt{jalis}, \texttt{jalrs}, and \texttt{jmprt} instructions form the basis of SIMD context control in the HARP instruction set.
Context is created using \texttt{clone}, the waiting threads are spawned using \texttt{jalrs} or \texttt{jalis}, ``jump-and-link immediate/register and spawn'', and finally the parallel section returns using \texttt{jmprt}, ``jump register and terminate'', best thought of as ``return and terminate.''

\section*{Assembler}
The assembler assembles a .s assembly file into a .HOF object file.\\
It can be given an optional archID parameter to control the architecture of the assembled object file.
\begin{verbatim}
harptool -A -o output.HOF input.s
\end{verbatim}

\section*{Linker}
The linker combines HOF files and produces raw RAM images for use by the emulator.\\
An intended future use is the conversion of multiple HOF files to statically-linked HOF executables.\\
It also takes an optional archID parameter.
\begin{verbatim}
harptool -L -o output.bin obj1.HOF obj2.HOF obj3.HOF
\end{verbatim}

\section*{Emulator}
The emulator emulates a HARP system. It is provided with the .bin containing the memory state to be run and an optional archID parameter.
\begin{verbatim}
  harptool -E -c example.bin
\end{verbatim}

\section*{Disassembler}
The disassembler is used to convert HOF files to equivalent assembly files.
One of its intended uses is the conversion of HOF object files between different HARP ISAs, say from 8w32/32 to 8b32/32.


\section{Interrupts}
The HARP interrupt mechanism is simple.
For SIMD lane 0, there is a shadow register file, program counter, and active lane count.
When an interrupt occurs, the state of lane zero is saved into these shadow registers, and execution resumes at the kernel entry point.
The type of interrupt is specified by the value placed in register 0 at this time, according to the following table:

\begin{center}
\begin{tabular}{cl}
\textbf{Number}&\textbf{Description}\\
\hline
0    &Trap (user-generated interrupt)\\
1    &Page fault due to absence from TLB\\
2    &Page fault due to permission violation\\
3    &Invalid/unsupported instruction\\
4    &Divergent branch\\
5    &Numerical domain (divide by zero)\\
6-7&(reserved for future exceptions)\\
8    &Console input\\
\end{tabular}
\end{center}

The first eight interrupt numbers are reserved for internal CPU-generated exceptions, and all of the remaining numbers are free for use by hardware.


%\begin{center}
%\includegraphics{fig/simd_core}
%\end{center}



\section{Default I/O Devices}
The emulator currently only supports a single I/O device, simple console I/O.
Writing to the address \texttt{0x800...0} (an address with its MSB set and all other bits cleared) causes text to be written to the display.
Input on this console interface causes an interrupt (number 8).

\end{document}
